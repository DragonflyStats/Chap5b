\section{Case Deletion Diagnostics for LME models}

\citet{HaslettDillane} remark that linear mixed effects models
didn't experience a corresponding growth in the use of deletion
diagnostics, adding that \citet{McCullSearle} makes no mention of
diagnostics whatsoever.

\citet{Christensen} describes three propositions that are required
for efficient case-deletion in LME models. The first proposition
decribes how to efficiently update $V$ when the $i$th element is
deleted.
\begin{equation}
V_{[i]}^{-1} = \Lambda_{[i]} - \frac{\lambda
\lambda\prime}{\nu^{}ii}
\end{equation}


The second of christensen's propostions is the following set of
equations, which are variants of the Sherman Wood bury updating
formula.
\begin{eqnarray}
X'_{[i]}V_{[i]}^{-1}X_{[i]} &=& X' V^{-1}X -
\frac{\hat{x}_{i}\hat{x}'_{i}}{s_{i}}\\
(X'_{[i]}V_{[i]}^{-1}X_{[i]})^{-1} &=& (X' V^{-1}X)^{-1} +
\frac{(X' V^{-1}X)^{-1}\hat{x}_{i}\hat{x}' _{i}
(X' V^{-1}X)^{-1}}{s_{i}- \bar{h}_{i}}\\
X'_{[i]}V_{[i]}^{-1}Y_{[i]} &=& X\prime V^{-1}Y -
\frac{\hat{x}_{i}\hat{y}' _{i}}{s_{i}}
\end{eqnarray}


\citet{schabenberger} examines the use and implementation of
influence measures in LME models.

Influence is understood to be the ability of a single or multiple
data points, through their presences or absence in the data, to
alter important aspects of the analysis, yield qualitatively
different inferences, or violate assumptions of the statistical
model \citep{schabenberger}.

Outliers are the most noteworthy data points in an analysis, and
an objective of influence analysis is how influential they are,
and the manner in which they are influential.

\citet{schabenberger} describes a simple procedure for quantifying
influence. Firstly a model should be fitted to the data, and
estimates of the parameters should be obtained. The second step is
that either single of multiple data points, specifically outliers,
should be omitted from the analysis, with the original parameter
estimates being updated. This is known as `leave one out \ leave k
out' analysis. The final step of the procedure is comparing the
sets of estimates computed from the entire and reduced data sets
to determine whether the absence of observations changed the
analysis.



A residual is the difference between an observed quantity and its
estimated or predicted value. In LME models, there are two types
of residuals, marginal residuals and conditional residuals. A
marginal residual is the difference between the observed data and
the estimated marginal mean. A conditional residual is the
difference between the observed data and the predicted value of
the observation. In a model without random effects, both sets of
residuals coincide.

\citet{schabenberger} notes that it is not always possible to
derive influence statistics necessary for comparing full- and
reduced-data parameter estimates. \citet{HaslettDillane} offers an
procedure to assess the influences for the variance components
within the linear model, complementing the existing methods for
the fixed components. The essential problem is that there is no
useful updating procedures for $\hat{V}$, or for $\hat{V}^{-1}$.
\citet{HaslettDillane} propose an alternative , and
computationally inexpensive approach, making use of the
`delete=replace' identity.

\citet{Haslett99} considers the effect of `leave k out'
calculations on the parameters $\beta$ and $\sigma^{2}$, using
several key results from \citet{HaslettHayes} on partioned
matrices.






In LME models, fitted by either ML or REML, an important overall
influence measure is the likelihood distance \citep{cook82}. The
procedure requires the calculation of the full data estimates
$\hat{\psi}$ and estimates based on the reduced data set
$\hat{\psi}_{(U)}$. The likelihood distance is given by
determining


\begin{eqnarray}
LD_{(U)} &=& 2\{l(\hat{\psi}) - l( \hat{\psi}_{(U)}) \}\\
RLD_{(U)} &=& 2\{l_{R}(\hat{\psi}) - l_{R}(\hat{\psi}_{(U)})\}
\end{eqnarray}


\addcontentsline{toc}{section}{Bibliography}

\bibliography{transferbib}
\end{document}