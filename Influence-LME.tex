\documentclass[00-ResidualsMain.tex]{subfiles}
\begin{document}
	
	%---------------------------------------------------------------%
	
	
	\subsection*{Influence Analysis for LME Models} %1.1.3
	The linear mixed effects model is a useful methodology for fitting a wide range of models. However, linear mixed effects models are known to be sensitive to outliers. Christensen et al advises that identification of outliers is necessary before conclusions may be drawn from the fitted model.
	
	Standard statistical packages concentrate on calculating and testing parameter estimates without considering the diagnostics of the model.The assessment of the effects of perturbations in data, on the outcome of the analysis, is known as statistical influence analysis. Influence analysis examines the robustness of the model. Influence analysis methodologies have been used extensively in classical linear models, and provided the basis for methodologies for use with LME models.
	Computationally inexpensive diagnostics tools have been developed to examine the issue of influence (Zewotir).
	Studentized residuals, error contrast matrices and the inverse of the response variance covariance matrix are regular components of these tools.
	
	\subsection*{Influence Statistics for LME models} %1.1.4
	Influence statistics can be coarsely grouped by the aspect of estimation that is their primary target:
	\begin{itemize}
		\item overall measures compare changes in objective functions: (restricted) likelihood distance (Cook and Weisberg 1982, Ch. 5.2)
		\item influence on parameter estimates: Cook's  (Cook 1977, 1979), MDFFITS (Belsley, Kuh, and Welsch 1980, p. 32)
		\item influence on precision of estimates: CovRatio and CovTrace
		\item influence on fitted and predicted values: PRESS residual, PRESS statistic (Allen 1974), DFFITS (Belsley, Kuh, and Welsch 1980, p. 15)
		\item outlier properties: internally and externally studentized residuals, leverage
	\end{itemize}
	%---------------------------------------------------------------------------%
	
	\subsection*{Influence Analysis for LME Models} %1.1.3
	The linear mixed effects model is a useful methodology for fitting a wide range of models. However, linear mixed effects models are known to be sensitive to outliers. Christensen et al. advises that identification of outliers is necessary before conclusions may be drawn from the fitted model.
	
	Standard statistical packages concentrate on calculating and testing parameter estimates without considering the diagnostics of the model.The assessment of the effects of perturbations in data, on the outcome of the analysis, is known as statistical influence analysis. Influence analysis examines the robustness of the model. Influence analysis methodologies have been used extensively in classical linear models, and provided the basis for methodologies for use with LME models.
	Computationally inexpensive diagnostics tools have been developed to examine the issue of influence (Zewotir).
	Studentized residuals, error contrast matrices and the inverse of the response variance covariance matrix are regular components of these tools.
	
	\subsection*{Influence Statistics for LME models} %1.1.4
	Influence statistics can be coarsely grouped by the aspect of estimation that is their primary target:
	\begin{itemize}
		\item overall measures compare changes in objective functions: (restricted) likelihood distance (Cook and Weisberg 1982, Ch. 5.2)
		\item influence on parameter estimates: Cook's  (Cook 1977, 1979), MDFFITS (Belsley, Kuh, and Welsch 1980, p. 32)
		\item influence on precision of estimates: CovRatio and CovTrace
		\item influence on fitted and predicted values: PRESS residual, PRESS statistic (Allen 1974), DFFITS (Belsley, Kuh, and Welsch 1980, p. 15)
		\item outlier properties: internally and externally studentized residuals, leverage
	\end{itemize}
\end{document}